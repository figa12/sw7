The focus of this project was to make an application that would improve the experience involving both hosting and attending an exhibition.
This is defined in our problem definition in \chapref{chap:intro}:
\begin{quote}
\textit{"How can we ease the creation of exhibitions, while enhancing the user's experience by providing them with relevant information while at the exhibition."}
\end{quote}
In order to give each user a customizable experience, we needed to be able to uniquely identify each user at an exhibition. We chose to do this using \ac{nfc} tags because they allow us to register when a user scans them and thereby providing them with a user ID.
 
With the identification technology in place we started focusing on our system. The system itself is split into two parts, a website and a mobile application.

The website is a tool for the exhibition organizers, with the purpose of creating and editing exhibitions. They are able to set up an exhibition and its floor plan. This allows them to plot in the location of booths and the walk paths around the exhibition. Each booth has a company and one or more categories assigned to them which is used on the mobile application to allow the user to specify what type of booths he is interested in. The organizers can also create feeds that are related to a specific booth and create events for a global schedule that is related to the entire exhibition. The creation of feeds is only implemented with limited functionality. Each booth should be able to create their own feeds and schedule entries, however this would require a login system which is not yet implemented. The organizers are also able to load an old exhibition and edit it.

The mobile application is implemented on the Android platform. The first time a user scans an \ac{nfc} tag they are asked to select which booths they want to subscribe to, the booths belong to different categories, accordingly to the categories they were assigned by the organizer. After the user has signed up for an exhibition they are taken to the tab activity where they can navigate between four tabs; ``Info'', ``Feeds'', ``Schedule'', and ``Floor plan''.

The ``Info'' tab provides the user with information about the exhibition, such as the name and a description. The ``Feeds'' tab shows the user feeds from the different booth they have signed to, if the user wants to change their subscriptions this is also possible, allowing them to fully customize which booths they want to receive feeds from. The user can also press a feed and get a pop-up with the full description of that feed. The ``Schedule'' tab shows a list of different events happening at the exhibition. The user will only receive events from the booths they have subscribed to. The ``Floor plan'' shows the floor plan that the organizers created with the tool, it also shows all the booths and walk paths around the exhibition. If a user scans a tag on a booth with the application closed, the application with open on the ``Floor plan'' tab and snap to the booth and a red dot will show up on the map to indicate where the user is. They are also able to press another booth on the floor plan and navigate to it by pressing the button. 

If the user arrives at a new exhibition and scan another \ac{nfc} tag, they will sign out of the previous exhibition and sign up for the new exhibition and be prompted to choose new booths. 