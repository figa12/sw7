The focus of this project was to make an application that would improve the experience involving both organisers and attendees at an exhibition.
This is defined in our problem definition in \chapref{chap:intro}:
\begin{quote}
\textit{"How can we ease the creation of exhibitions, while enhancing the user's experience by providing them with relevant information while at the exhibition."}
\end{quote}
In order to give each user a customizable experience, the first problem that had to be solved was how the users should be identified at an exhibition. We chose to do this using \ac{nfc} tags because they allow us to register when a user scans them and thereby providing them with a unique user ID.
 
With the identification technology in place we started focusing on the rest of the system. The system itself is split into three parts, a website, a server and a mobile application.

The website is a tool for the exhibition organisers, with the purpose of creating and editing exhibitions. They are able to set up an exhibition and its floor plan. This allows them to plot in the location of booths and the walk paths around the exhibition. Each booth has a company and a categories assigned to them, which is used on the mobile application to allow the user to specify what type of booths they are interested in. The organisers can also create feeds that are related to a specific booth and create events for a global schedule that is related to the entire exhibition. We managed to implement all the features for the website mentioned in \secref{sec:exhibsystem}. However, the creation of feeds is only implemented with limited functionality. Each booth should be able to create their own feeds and schedule entries, however this would require a login system which we decided not to focus on for this project. The organisers and the booth organisers can still create feeds through the website for a specific booth, but there is no control of who creates feeds for which booth. The organisers are also able to load an old exhibition and edit it.

The mobile application is implemented on the Android platform. The first time a user scans an \ac{nfc} tag they are asked to select which booths they want to subscribe to. Each booth belong to different categories, assigned by the exhibition organizer. After the user has signed up for an exhibition they are taken to the tab activity where they can navigate between four tabs; ``Info'', ``Feeds'', ``Schedule'', and ``Floor plan''.

The ``Info'' tab provides the user with information about the exhibition, such as the name and a description. The ``Feeds'' tab shows the user feeds from the different booth they have subscribed to, the user can, if wanted, change their subscriptions, allowing them to fully customize which booths they want to receive feeds from. The user can also click a feed and get a pop up with the full description of that feed. The ``Schedule'' tab shows a list of different events happening at the exhibition. The user will only receive events from the booths they are subscribed to. Each schedule entry also has a time counter showing how long until the event. The ``Floor plan'' shows the floor plan that the organisers created with the tool, it also shows all the booths and walk paths around the exhibition. If a user scans a tag on a booth with the application closed, the application will open on the ``Floor plan'' tab and snap to the booth. A red dot is shown on the floor plan to indicate where the booth is placed. They are also able to click another booth on the floor plan and navigate to it by clicking the button. We managed to implement all features for the mobile application mentioned in \secref{sec:exhibsystem} except for one. The user is not able to browse for exhibitions or browse recently added exhibitions.

If the user arrives at a new exhibition and scan another \ac{nfc} tag, they will sign out of the previous exhibition and sign up for the new exhibition and be prompted to choose new booths. 

Based on the tests performed in \chapref{chap:tests} we can guarantee that the mobile application is working as intended. We have not made any graphical user interface tests, to ensure the design of our mobile application and website. We have managed to create a system for helping organisers create exhibitions while also enhancing the user experience for attendees at an exhibition, which was the of the project.