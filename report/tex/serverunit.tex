\subsection*{Serverside Unit Testing}
We have performed unit testing on the serverside api using an open source tool called PHPUnit\citep{phpunit} which is made for easy and simple unit testing of PHP code. The basic approach of using PHPUnit is to create a new file that includes the functions that you want to test and has a new class that extends \lstinline|PHPUnit_Framework_TestCase|. All public methods in this new class that are named with a prefix of \lstinline|test| will be included in the test. Inside each of these methods all kinds of assertions can be made, an example could be \lstinline|$this->assertNotNull($var)|.


We have performed unit tests on all the functions that our server side API uses. These unit tests were done using an open source tool called PHPUnit\citep{phpunit}. All calls to the api must return a result of the correct format, this is checked for every function, see \autoref{lst:checkTemplate}. \lstinline{checkTemplate} contains four asserts to validate the format of the result.
 
\begin{lstlisting}[numbers=none, basicstyle=\ttfamily]
Time: 616 ms, Memory: 2.75Mb
OK (12 tests, 52 assertions)
\end{lstlisting}

\begin{lstlisting}[language=phpstyle, label=lst:checkTemplate, caption={checkTemplate unit test}]
private function checkTemplate($testResult, $requestCode) {
    // Is the json valid
    $this->assertNotNull($testResult, "Are you missing a Type or RequestCode?");
    // Does RequestCode exist?
    $this->assertTrue(property_exists($testResult, "RequestCode"));
    // Does Data exist?
    $this->assertTrue(property_exists($testResult, "Data"));    
    // RequestCode must be the same as the one given.
    $this->assertEquals($testResult->RequestCode, $requestCode);
}
\end{lstlisting}%$

\begin{lstlisting}[language=phpstyle, label=lst:testCreateUser, caption={createUser unit test}]
public function testCreateUser() {
    $_POST['RequestCode'] = 1;
    $_POST['ExhibId'] = $this->exhibId;
    $_POST['Type'] = "CreateUser";

    $testResult = init();
    $testResult = json_decode($testResult);

    $this->checkTemplate($testResult, 1);

    $this->assertNotEmpty($testResult->Data);
    
    return $testResult->Data->userId;      
}
\end{lstlisting}%$

\begin{lstlisting}[language=phpstyle, label=lst:testCheckFeeds, caption={checkFeeds unit test}]
/** @depends testCreateUser */
public function testCheckFeeds($userId) {
    $now = time();
    $_POST["Type"] = "CheckFeeds";
    $_POST["RequestCode"] = 1;
    $_POST['UserId'] = $userId;
    $_POST['TimeStamp'] = $now;

    $testResult = init();
    $testResult = json_decode($testResult);

    $this->checkTemplate($testResult, 1);

    $numFeeds = $testResult->Data[0];
    $numFeeds = $numFeeds->num;

    $this->assertNotNull($numFeeds);

    return array($numFeeds, $now, $userId);
}

/** @depends testCheckFeeds */
public function testCheckFeeds2($args) {
    $numFeeds = $args[0];
    $now = $args[1];
    $userId = $args[2];

    $this->createFeed($now);
    
    $newArgs = $this->testCheckFeeds($userId);
    $newNum = $newArgs[0];
    
    $this->assertEquals($numFeeds + 1, $newNum);
}
\end{lstlisting}
%%% Local Variables: 
%%% mode: latex
%%% TeX-master: "../master"
%%% End: 
