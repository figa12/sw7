\section{ServerSyncService}

To communicate with the server from the application we have created a class called \lstinline|ServerSyncServices| that manages all server communication. The class inherits from the abstract class \lstinline|AsyncTask| which makes it easy to run a task on a new thread and send the results back to the UI thread. We must override the abstract method \lstinline|doInBackground()| which is what is run on a new thread, and the result of the computation is send to \lstinline|onPostExecute(result)|.

\lstinline|ServerSyncService| sends a request to the server, parses the response, and sends the result to the correct receiver. Every request consists of a \lstinline|"RequestCode"| and a \lstinline|"Type"|. The value of \lstinline|"RequestCode"| is a unique static final integer which the server sends back as part of the result, we use it in the \lstinline|ServerSyncService| to know how to process the response and where to send the result. The value of \lstinline|"Type"| is the actual request we make to the server. Each type may also have some parameters. \todo{Det står måske også i afsnittet før}

\autoref{lst:serverrequest} shows how to make a request to the server. This example is a request made from the \lstinline|ExhibitionInfoFragment|.

\lstinline|BasicNameValuePair| is a class that consists of a name and a value bound to that name. \todo{Skriv måske hvordan det egentlig bliver læst på serveren}

\begin{lstlisting}[language=java, label=lst:serverrequest, caption={Get exhibition information request}]
new ServerSyncService(super.getContext()).execute(
        new BasicNameValuePair("RequestCode", String.valueOf(ServerSyncService.GET_EXHIBITION_INFO)),
        new BasicNameValuePair("Type", "GetExhibitionInfo"),
        new BasicNameValuePair("ExhibId", String.valueOf(this.tabActivity.getExhibId());
\end{lstlisting}

\begin{description}
\item[Line 1] Create a new \lstinline|ServerSyncService|. We give it the context as argument so the response can be passed to the context. \lstinline|execute()| is the method that starts the request from the server, where the next lines are the arguments.
\item[Line 2] A name/value pair with the name \lstinline|"RequestCode"| and the value is a unique integer called \lstinline|GET_EXHIBITION_INFO| which we parse to a string.
\item[Line 3] A name/value pair with the name \lstinline|"Type"| and the actual request called \lstinline|"GetExhibitionInfo"| which is the request used to display information about the exhibition in the info tab.
\item[Line 4] For this type of request the server expects the parameter name/value pair \lstinline|"ExhibId"|. The exhibtion id is saved in the \lstinline|TabActivity|.
\end{description}
\autoref{lst:doinbackground} is our implementation of the abstract method \lstinline|doInBackground()|. When \lstinline|execute()| is run it creates a new thread for us and runs our implementation of \lstinline|doInBackground()| on that thread with the provided \lstinline|NameValuePair| array.

\begin{lstlisting}[language=java, label=lst:doinbackground, caption={The async abstract method \lstinline|doInBackground()|}]
@Override
protected String doInBackground(NameValuePair... pairs) {
    HttpParams httpParameters = new BasicHttpParams();

    HttpConnectionParams.setConnectionTimeout(httpParameters, 15000);
    HttpConnectionParams.setSoTimeout(httpParameters, 15000);

    HttpClient httpclient = new DefaultHttpClient(httpParameters);
    
    HttpPost httppost = new HttpPost(this.serverUrl);
    httppost.setEntity(new UrlEncodedFormEntity(Arrays.asList(pairs)));

    HttpResponse response = httpclient.execute(httppost);

    HttpEntity entity = response.getEntity();
    return EntityUtils.toString(entity);
}
\end{lstlisting}

\begin{description}
\item[Line 3] Get some basic http parameters.
\item[Lines 5-6] Set timeouts on the http parameters. Allow 15 seconds to create a connection, and allow 15 seconds to to get a response.
\item[Line 8] Create the http client with our http parameters.
\item[Lines 10-11]create an \lstinline|HttpPost| object with the server address. Convert our name/value pairs to \lstinline|UrlEncodedFormEntity| and set the entity as the request.
\item[Line 13] Executes the http request on the http client and saves the response. This pauses the thread until a response is received.
\item[Lines 15-16] Get the response as an entity, convert the entity to a string, and return it.
\end{description}
After \lstinline|doInBackground()| returns the result, the \lstinline|AsyncTask| automatically passes the result to \lstinline|onPostExecute()|. \autoref{lst:onpostexecute} shows our implementation of this method and how we treat the response from the server.

\begin{lstlisting}[language=java, label=lst:onpostexecute, caption={The async method \lstinline|onPostExecute()|}]
@Override
protected void onPostExecute(String result) {
    if (result == null || result.equals("")) {
        Log.e(ServerSyncService.class.getName(), "No connection found-ish.");
        return;
    } else if (result.equals("Could not complete query. Missing type") || result.equals("Missing request code!")) {
        Log.e(ServerSyncService.class.getName(), result);
        return;
    } else if(result.equals("Timeout")) {
        Log.e(ServerSyncService.class.getName(), result);
        return;
    }

    InputStream stream = new ByteArrayInputStream(result.getBytes("UTF-8"));
    this.readJsonStream(stream);
    stream.close();
}
\end{lstlisting}\todo{Der skal stå et sted at requestCode burde være det første Json object i stedet}

\begin{description}
\item[Lines 3-5] If the result is empty, then we did not find the server.
\item[Lines 6-8] If the server were missing the type parameter or the request code, then it could not process the request.
\item[Lines 9-11] If the timeout limit was reached. We print the error in the log. \todo{It would be a good idea to handle these kinds of errors in a better way, like letting the user know that a connection was not established and maybe allowing him to try again.}
\item[Lines ] ... \todo{do}
\end{description}