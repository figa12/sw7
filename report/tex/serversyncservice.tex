\section{ServerSyncService}
text\todo{}

\begin{lstlisting}[language=java, label=lst:serverrequest, caption={Get exhibition information request}]
new ServerSyncService(super.getContext()).execute(
                new BasicNameValuePair("RequestCode", String.valueOf(ServerSyncService.GET_EXHIBITION_INFO)),
                new BasicNameValuePair("Type", "GetExhibitionInfo"),
                new BasicNameValuePair("ExhibId", String.valueOf(this.tabActivity.getExhibId());
\end{lstlisting}

\textit{ServerSyncService} implements the abstract class \textit{AsyncTask}.\todo{}

\begin{lstlisting}[language=java, label=lst:doinbackground, caption={The async abstract method \textit{doInBackground}}]
@Override
protected String doInBackground(NameValuePair... pairs) {
    //Create the HTTP request
    HttpParams httpParameters = new BasicHttpParams();

    //Setup timeouts
    HttpConnectionParams.setConnectionTimeout(httpParameters, 15000);
    HttpConnectionParams.setSoTimeout(httpParameters, 15000);

    HttpClient httpclient = new DefaultHttpClient(httpParameters);
    HttpPost httppost = new HttpPost(serverUrl);

    httppost.setEntity(new UrlEncodedFormEntity(Arrays.asList(pairs)));

    HttpResponse response = httpclient.execute(httppost);
    HttpEntity entity = response.getEntity();
    
    return EntityUtils.toString(entity);
}
\end{lstlisting}\todo{Find ud af hvad de forskellige klasser de gør}

\begin{lstlisting}[language=java, label=lst:onpostexecute, caption={The async method \textit{onPostExecute}}]
@Override
protected void onPostExecute(String result) {
    if (result == null || result.equals("")) {
        Log.e(ServerSyncService.class.getName(), "No connection found");
        return;
    } else if (result.equals("Could not complete query. Missing type") || result.equals("Missing request code!")) {
        Log.e(ServerSyncService.class.getName(), result);
        return;
    } else if(result.equals("Timeout")) {
        Log.e(ServerSyncService.class.getName(), result);
        return;
    }

    int objectIndex = result.indexOf('['); // always contains Json object

    int requestCode = Integer.valueOf(result.substring(0, objectIndex));
    result = result.substring(objectIndex); // cut away requestCode

    this.readJsonStream(new ByteArrayInputStream(result.getBytes("UTF-8")), requestCode);
}
\end{lstlisting}\todo{Der skal stå et sted at requestCode burde være det første Json object i stedet}