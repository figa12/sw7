\todo{Skal bindes sammen med de tidligere afsnit hvor vi har snakket om website og om vi har fået implementeret det vi ville og evt hvad der er anderledes}As mentioned in \secref{sec:problemstatement}, the purpose of the website is to allow the exhibition managers to make a floor plan, booths, schedule and other concepts of an exhibition. We chose to design it as an (almost) one-page site. 
\todo{Beskrivelse af website. En gennemgang af det, hvad kan man med website. Mere beskrivelse af de forskellige tabs osv}
The website is built using HTML, PHP and jQuery. The main tab on the website is the "Map" tab, which is used for designing the floor plan. This tab is built using the Google Maps API. Using this tab the organizer can create a complete exhibition with booths etc.
After creating an exhibition the user can add feeds to the exhibition, using the "Feeds" tab. A schedule can be added using the the schedule tab.
It is also possible to load an existing exhibition from the database, for the loaded exhibition the floor plan will be shown but the user is unable to edit the floor plan in the current version.

The website uses the AJAX technology for form submit to make the website more user-friendly. The AJAX posts to PHP, which handles the database interaction. When connecting to the database we are using prepare statements, to make sure no malicious data is inserted into the database.

The design of the website is done using Twitter Bootstrap making it more slick, and uniform.
The main focus in the website is functionality and therefore the design and usability is not the main priority.




Flowchart diagram. \todo{Tænker flow chart skal være først}


Schedule - booth with same name, doesn't \todo{Doesn't?}