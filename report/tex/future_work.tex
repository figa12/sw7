Due to time constraints in our project we have not managed to implement all functionalities. 

\begin{description}
%\item[Other map items] We only show booths and paths on the map. It is not possible to show a stage or a 'you are here' stand. There is a workaround for this at the moment, you can create a booth instead, but it might confuse the exhibition organisers.
\item[Exhibition schedule] The exhibition organisers cannot create a schedule bound to the exhibition, you can only create a schedule for a booth.% This is also part of the lack of functionality mentioned above.
\item[Exhibition statistics] Right now we save how many times each user has visited a booth, i.e. each time they scan an \ac{nfc} tag at a booth. All the statistics should be shown to the exhibition organisers.

Metrics could be: How many subscribers each booth has, how many times navigation to a booth has been requested, and how many users an exhibition has.
\item[User recommendations] Based on the users subscriptions and which booths the user has visited, we could recommend other booths which belongs to the same category.
\item[Tag scanned event] When an \ac{nfc} tag is scanned we could provide the user with different actions, such as: Navigate from this booth to another, subscribe/unsubscribe from booth, or you could show the booth on floor plan.
\item[Website login system] On the website a login system should be implemented, so make sure exhibition managers can only manage their own exhibitions. It should also provide the option for individual booth managers to login, creating both feeds and schedule for the booths they manage.
\item[Automatic \ac{nfc} creation] From the website, when finishing creating an exhibition, a list should be provided to the user, containing all information for the \ac{nfc} tags of the exhibition. This list could then be sent to a supplier, and the user would receive pre made \ac{nfc} tags.
\item[Automatic tile creation] From the website, the user should have the possibility to upload a single picture floor plan, and from this picture the server would create tiles for the floor plan to use. This would cut out the need for the developers to create the tiles. This feature should make it possible to change the background tiles on the website, when creating a floor plan. This change should also be reflected in the database, where an tileset field should be added to the "exhibs" table.
\item[Booth queue system] It could be useful if each booth had a queue system so the visitors could see if they would have to wait in line for a specific booth, but that requires the booth to manually update the queue times. This system could also be implemented as a virtual queue so that visitors could sign up for events without actually having to stand in line.
%\item[Booth options] When you click on a booth on the floor plan and subscribe / unsubscribe from that booth.
\item[Recent exhibitions] Some kind of menu where the user can see their recently scanned exhibitions so it would be possible to be navigate between different exhibitions. 
\item[Active exhibitions] The organisers should be allowed to close an exhibition, so when the user opens the application it should perform a check to see if this exhibition is still active.
\item[Non \ac{nfc} support] We realize that not all phones have \ac{nfc}, so support for other technologies such as \acp{qr} could be implemented.
\item[Usability test] A usability test of our application and website should be performed in order to find the possible flaws in the interface design or document that our interfaces are user-friendly.
\item[Website system interface] At the moment the website communicates directly with the database, the part of the website that writes to the database should be in its own component. This way it will be split up in more components, making it easier to maintain.
\end{description}

%\item Gem activity i baggrunden, virker ikke pga. fragment manager griseri
%\item android unit tests
