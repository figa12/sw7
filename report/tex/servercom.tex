\chapter{Server communication}
This chapter describes the communication between server and client, aswell as the design choices and
implementation of database. \todo{Bør database være et kapitel for sig selv?}

\section{Communication}
\label{sec:com}

The communication betweeen the client application and the server is done by HTTP POST requests.
The client sends a HTTP POST request to the server containing a certain list of parameters which
define what the server is supposed to do. All requests must take atleast two required parameters, a \textit{RequestCode} and a
\textit{Type}.

\textit{RequestCode} is an integer that is simply passed through the server, it is not handled in any
way on the server. The purpose of the RequestCode parameter is to distinguish the request on the client allowing
it to execute the appropriate function.

\textit{Type} is used to identify the request. A typical request is to get a list of feeds, and the
value of the \textit{Type} parameter would be ``GetFeeds'' in this case.\\

Depending on the \textit{Type} of the request, additional parameters may be required.\\

As an example, \textit{GetFeeds} takes the following parameters:
\begin{itemize}
\item RequestCode
\item UserId
\item Limit
\end{itemize}

The approach is to process the request on the serverside, and make the right calls to the
database. The server will then return a JSON object containing all the relevant information that was
requested. This JSON object can easily be parsed on the client side. The reason why we chose JSON as
a format is because it is well supported and easy to generate in both android and PHP.

\begin{lstlisting}[language=php, caption=getFeeds function call]
$requestCode = $_POST["RequestCode"];

switch ($_POST['Type']) {
        case "GetFeeds":
            getFeeds($con, $_POST['UserId'], $_POST['Limit']);
            break;
}
\end{lstlisting}

\begin{description}
\item \textbf{Line 1: }Get the \textit{RequestCode} parameter.
\item \textbf{Lines 3-7: }If \textit{Type} equals ``GetFeeds'', call the getFeeds function.
\end{description}

\todo{Nævn de andre cases, måske her?}

\begin{lstlisting}[language=php, caption=getFeeds function]
function getFeeds($con, $userId, $limit) {

    global $requestCode;

    #Escape special characters to avoid SQL injection attacks
    $userId    = $con->real_escape_string($userId);
    $limit     = $con->real_escape_string($limit);

    $query =   "SELECT feeds.id, feeds.boothid, feeds.header, feeds.description, feeds.feedtime, userbooths.userid, companies.logo, booths.name ".
        "FROM feeds ".
        "LEFT JOIN userbooths ".
        "ON feeds.boothid = userbooths.boothid ".
        "LEFT JOIN booths ".
        "ON feeds.boothid = booths.id ".
        "LEFT JOIN companies ".
        "ON booths.companyid = companies.id ".
        "WHERE userid = ".$userId." ".
        "AND sub = 1 ".
        "ORDER BY feeds.feedtime DESC ".
        "LIMIT "$limit;

    $result = $con->query($query);

    $json = array();
    while($row = $result->fetch_object()) {
      array_push($json, $row);
    }
    echo $requestCode;
    echo json_encode($json);
}
\end{lstlisting}

\begin{description}
\item \textbf{Line 1: }The function takes three parameters, the mysqli object which has a connection
  established to the database, the user id from the client, and the limit which is also given by the client.
\item \textbf{Line 3: }Since the \textit{RequestCode} is always a required parameter the variable is global.
\item \textbf{Lines 6-7: }Escape special characters to avoid malicious SQL injections.
\item \textbf{Lines 9-20: }Create the SQL query with the user id and the limit paramter.
\item \textbf{Line 22: }Execute the query.
\item \textbf{Lines 24-29: }Get the results from the database and push them to the json array. When
  all of the results have been pushed to the array, the \textit{RequestCode} will be echoed and then
  a json encoded version of the json array.
\end{description}

\section{Database}
We are using a MySQL database with the MyISAM storage engine and which is located at a remote server. \todo{info om hvor osv.?}

The server is to contain all data about each exhibition, the companies at the exhibition, the feeds, the schedule, and some basic information about the users.

\begin{figure}[H]
\centering
\includegraphics[page=1,width=1\linewidth]{img/sw7ERD.pdf}
\caption{Entity-relation diagram}
\label{fig:erd}
\end{figure}

%%% Local Variables: 
%%% mode: latex
%%% TeX-master: "../master"
%%% End: 
