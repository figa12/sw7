\section{Black-box testing}
We have created some black-box test cases for our Android application. We are performing the tests ourselves, we do this in order to test that all the different functionality requirements are working and producing correct results.

All test cases are presented in the form of a case header consisting of the case name, objective of the test, preconditions for the test, and the requirement we are satisfying. After the header we have made a procedure describing the steps that should be performed in the case and for each step we have a success criteria describing the expected outcome of the step.

\newcommand{\testcase}[4]
{
\subsubsection*{#1}
\begin{center}
\begin{tabular}{p{1.8cm} p{8.52cm}}
\hline
\textbf{Case name} & #1\\
\textbf{Objective} & #2\\
\textbf{Preconditions} & #3\\
\textbf{Requirement} & #4\\
\hline
\\
\end{tabular}
\end{center}
}

\newcommand{\casetwo}{Load old feeds}
\newcommand{\casethree}{Scan while reading a feed}
\newcommand{\casefour}{Load new feeds}

\testcase
{\casetwo}
{Make sure the application does not stall when scrolling to the bottom of the feed list}
{Connected to the internet, tab activity open}
{Save phone memory}

\begin{center}
\begin{tabular}{| c | p{4.6cm} | p{4.6cm} |}
\hline
\textbf{Step} & \textbf{Procedure} & \textbf{Success Criteria}\\
\hline
1 & Click the \textit{Feeds} tab & \textit{Feed} tab is displayed\\
\hline
2 & Scroll to the bottom of the list & Older feeds are loaded from the server\\
\hline
\multicolumn{3}{c}{} \\% tom multicolumn for at få en snyde linjeafstand
\end{tabular}
\end{center}

\testcase
{\casethree}
{The feed window should close and show the booth on the map}
{Connected to the internet, tab activity open}
{Foreground mode}

\begin{center}
\begin{tabular}{| c | p{4.6cm} | p{4.6cm} |}
\hline
\textbf{Step} & \textbf{Procedure} & \textbf{Success Criteria}\\
\hline
1 & Click the \textit{Feeds} tab & \textit{Feed} tab is displayed\\
\hline
2 & Click a feed item & A window opens with the full feed descriptions\\
\hline
3 & Scan a booth \ac{nfc} tag & The feed window closes and the tab changes to the floor plan and snaps to a booth\\
\hline
\multicolumn{3}{c}{} \\% tom multicolumn for at få en snyde linjeafstand
\end{tabular}
\end{center}

\testcase
{\casefour}
{While the user is viewing feed list and a booth creates a new feed, the list should show that more feeds are available}
{Connected to the internet, tab activity open}
{}

\begin{center}
\begin{tabular}{| c | p{4.6cm} | p{4.6cm} |}
\hline
\textbf{Step} & \textbf{Procedure} & \textbf{Success Criteria}\\
\hline
1 & Click the \textit{Feeds} tab & \textit{Feeds} tab is displayed\\
\hline
2 & Wait until a new feed is created by a booth & A button appears in the top of the tab with the text "Click to load $X$ new items"\\
\hline
3 & Click the button in the top of the tab & $X$ new feed items are loaded at the top of the feed list\\
\hline
\multicolumn{3}{c}{} \\% tom multicolumn for at få en snyde linjeafstand
\end{tabular}
\end{center}

\newcommand{\casefive}{New feeds}
\newcommand{\casesix}{Register a user}
\newcommand{\caseseven}{Snap to booth}
\newcommand{\caseeight}{Navigate to booth}
\newcommand{\casenine}{Register new exhibition}

\testcase
{\casefive}
{Make sure the feed list is updated when new booths are chosen}
{Connected to the Internet, registered to an exhibition}
{Change booth subscriptions}

\begin{center}
\begin{tabular}{| c | p{4.6cm} | p{4.6cm} |}
\hline
\textbf{Step} & \textbf{Procedure} & \textbf{Success Criteria}\\
\hline
1 & Note the booth names from the feed list & \textit{None}\\
\hline
2 & Click the menu button & Menu opens\\
\hline
3 & Click the menu item \textit{Choose Categories} & Categories activity opens\\
\hline
4 & Change booth subscriptions and note the changes & \textit{None}\\
\hline
5 & Click the submit button & Categories activity closes\\
\hline
6 & Check that the feed list changed accordingly to your new subscriptions & Feeds changed successfully\\
\hline
\multicolumn{3}{c}{} \\% tom multicolumn for at få en snyde linjeafstand
\end{tabular}
\end{center}

\testcase
{\casesix}
{Make sure that when scanning without a registered user, you have to choose categories and scanning with a registered user or opening the application with a registered user, you do not}
{Connected to the Internet, \ac{nfc} is on}
{Register a user, booth subscriptions}

\begin{center}
\begin{tabular}{| c | p{4.6cm} | p{4.6cm} |}
\hline
\textbf{Step} & \textbf{Procedure} & \textbf{Success Criteria}\\
\hline
1 & Scan an \ac{nfc} tag & \textit{The application opens}\\
\hline
2 & Choose your categories and press submit & \textit{The tab activity opens}\\
\hline
3 & Exit the application & \\
4 & Open up the application again & The tab activity should open\\
\hline
\multicolumn{3}{c}{} \\% tom multicolumn for at få en snyde linjeafstand
\end{tabular}
\end{center}

\testcase
{\caseseven}
{The application should change tab to "Map" and snap to the booth that matches the boothID on the tag}
{Connected to the Internet,  tab activity open, \ac{nfc} is on}
{Snap to the booth the user is located at}

\begin{center}
\begin{tabular}{| c | p{4.6cm} | p{4.6cm} |}
\hline
\textbf{Step} & \textbf{Procedure} & \textbf{Success Criteria}\\
\hline
1 & Scan a booth \ac{nfc} tag& \textit{Change to "Map" tab and snap to booth}\\
\hline
\multicolumn{3}{c}{} \\% tom multicolumn for at få en snyde linjeafstand
\end{tabular}
\end{center}

\testcase
{\caseeight}
{The user should get the shortest path from where he last scanned a tag and to the booth he wants to navigate to}
{Connected to the Internet, be on the "Map" tab}
{Route planner}

\begin{center}
\begin{tabular}{| c | p{4.6cm} | p{4.6cm} |}
\hline
\textbf{Step} & \textbf{Procedure} & \textbf{Success Criteria}\\
\hline
1 & Press a booth on the floor plan & \textit{Booth info window should open}\\
\hline
2 & Press the "Navigate to" button & \textit{The shortest path should be drawn on the floor plan}\\
\hline
\multicolumn{3}{c}{} \\% tom multicolumn for at få en snyde linjeafstand
\end{tabular}
\end{center}

\testcase
{\casenine}
{End current exhibition session and start a new one with the new exhibID and prompt the user to choose categories}
{Connect to the Internet, tab activity open}
{Support for multiple exhibitions}
\begin{center}
\begin{tabular}{| c | p{4.6cm} | p{4.6cm} |}
\hline
\textbf{Step} & \textbf{Procedure} & \textbf{Success Criteria}\\
\hline
1 & Scan an \ac{nfc} tag & \textit{End current exhibition and open a new one}\\
\hline
2 & Choose categories for the new exhibition and press submit & \textit{The tab activity should open}\\
\hline
\multicolumn{3}{c}{} \\% tom multicolumn for at få en snyde linjeafstand
\end{tabular}
\end{center}

\subsection*{Case results}
\begin{description}
\item[\casetwo] This test was a success. The application does not stall and it loads the old feeds, until there are no more feeds to load
\item[\casethree] This test was a failure. When having a feed open and scanning an \ac{nfc} tag, nothing happened
\item[\casefour] This test was a success. When a booth creates a new feed, the button shows up and allows the user to manually load the new feed
\item[\casefive] This test was a success. When subscribing or unsubscribing to booths, their feeds are added and removed accordingly 
\item[\casesix] This test was a success. The first time the \ac{nfc} tag was scanned the application opened and we had to choose which booths to subscribe to, and after closing and opening the application again it opens up directly in the tab activity without having to choose categories
\item[\caseseven] This test was a success. When scanning an \ac{nfc} tag with a boothID the "Map" tab opened and snapped to the booth on the floor plan
\item[\caseeight] This test was a success. When clicking the booth the information box comes up and if  you press the "Navigate to" button a route is drawn from your last known location and to the booth
\item[\casenine] This test resulted in failure. When scanning a different \ac{nfc} tag the application crashes. If already signed up to an exhibition the application skips the exhibitionID on the tag and tries to snap to the booth, which boothID is on tag. This causes the application to crash if the boothID on the tag does not exist at the current exhibition.
\end{description}
