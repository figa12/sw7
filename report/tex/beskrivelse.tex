The basic idea is to map an exhibition and collect data from people's locations and interests. Furthermore the mobile application should aid the users to get more information from the exhibition.


\subsection*{Features}

\begin{itemize}
\item Application to load floorplan and make routes. (Administration).

\item Administration website\\
\textit{A place to insert information and location of booths.}
\end{itemize}

\subsection*{Mobile application}

\begin{itemize}
\item NFC tag reader

\item Should read a "global" tag that binds the application to the current exhibition.

\item Should also read a "local" tag that is bound to each booth.\\
\textit{This could trigger additional information about the booth, or help you get directions to anohter booth.}
\item Time schedule of events.

\item View floorplan.

\item Feed of realtime news. A booth could for example send a realtime post to all users interested in this booth.\\
\textit{Twitter could be incorporated in this.}

\item When the application is not bound to anything, the app should show an exhibition browser that lets you search for exhibitions that you might be interested in.
\end{itemize}

\subsection*{Ideas of ways to subscribe to booths}

\begin{itemize}
\item Every booth have some categories, and when the initial NFC is read, you can choose which categories you wish to subscribe to.

\item Every booth is subscribed to from start, and the user must manually unsubscribe from everything he finds uninteresting.
\end{itemize}

\subsection*{Information that could be gathered from the users/mobile app}

\begin{itemize}
\item Density / queues ( This could be shown to the users, realtime)

\item General interests

\item Statistics of each booth\\
\textit{When a user scans a tag at a booth, the hit should be recorded. An idea is that he chooses to either get more information or get directions to another booth. Data from both can be gathered.}\citep{rtsbog}
\end{itemize}
